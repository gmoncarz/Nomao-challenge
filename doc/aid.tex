%\documentclass[12pt]{article}
\documentclass[journal]{IEEEtran}
\usepackage[utf8]{inputenc}
\usepackage{listings}
\usepackage{lipsum}
\usepackage{graphicx}
\usepackage[spanish]{babel}
\usepackage{tikz}
\usepackage{pgf-pie}
\usepackage{babel,blindtext}
\usepackage{color}
\usepackage[stable]{footmisc}
\usepackage{supertabular}
\usepackage{longtable}



% Custom colors
\definecolor{deepblue}{rgb}{0,0,0.5}
\definecolor{deepred}{rgb}{0.6,0,0}
\definecolor{deepgreen}{rgb}{0,0.5,0}

% Default fixed font does not support bold face
\DeclareFixedFont{\ttb}{T1}{txtt}{bx}{n}{07} % for bold
\DeclareFixedFont{\ttm}{T1}{txtt}{m}{n}{06}  % for normal

\lstdefinestyle{python_gabo}{
  breaklines=true,
	language=Python,
	basicstyle=\ttm,
	otherkeywords={self},             % Add keywords here
	keywordstyle=\ttb\color{deepblue},
	emph={MyClass,__init__},          % Custom highlighting
	emphstyle=\ttb\color{deepred},    % Custom highlighting style
	stringstyle=\color{deepgreen},
	frame=tb,                         % Any extra options here
	showstringspaces=false,            % 
  commentstyle=\ttb,
}


\begin{document}

\title{Trabajo Final de AID}
\author{Moncarz, Gabriel}
\maketitle % this produces the title block


\begin{abstract}
TODO.
bla bla bla.
bla bla bla.
\end{abstract}

\begin{IEEEkeywords}
TODO
\end{IEEEkeywords}


\section{Introducción}

\subsection{Sobre ALRA/Nomao Challenge}
Nomao Challenge fue una competencia de Data Mining organizada por ALRA 
(Active Learning in Real-World Applications) y Nomao en el año 2012. Nomao 
\footnote{www.nomao.com} es un motor de busquedas de lugares, que colecta
información de lugares de diferentes fuentes (Web, celulares, tables, gps, 
etc). Esta información es almacenada en una base de datos interna. 
Cuando se realiza una consulta al motor de busqueda, éste debe retornar una
respuesta unificada. Una de las complejidades de devolver una
respuesta unificada, radica en el proceso de deduplicacion de datos. Este
proceso es el encargado de detectar si la información de 2 fuentes distintas
son asociadas a un mismo lugar o no. Por ejemplo, la tabla \ref{table:example1}
muestra los lugares que responden a la consulta "La poste" en francia. El
proceso de deduplicación debe identificar que el sitio 2 y 3 se refieren
al mismo lugar, pero el sitio 1 no.

\begin{table}[ht!]
\caption{Posibles lugares a retornar por una consulta}
\label{table:example1}
\centering
\begin{tabular}{l | l l l }
ID & Nombre & Direccion & Telefono  \\
\hline
1 & La poste & 13 Rue De La Clef 59000 Lille France & 3631 \\ 
2 & La poste & 13 Rue Nationale 59000 Lille France & 3631 \\
3 & La poste lille & 13 r. nationale 59000 lille & 0320313131 \\
\end{tabular}
\end{table}

Cada lugar provisto por un usuario es almacenada internamente. Se guarda
información sobre el nombre, dirección, geolocalizacion, página web,
telefono, fax, etc. Uno de los inconvenientes es que como los datos
provienen de fuentes distintas o a veces son tipeados manualmente,
lugares iguales pueden tener información distinta, como también
distintos lugares pueden tener algunos campos iguales, como por 
ejemplo el nombre.

El objetivo del desafio Nomao Challenge 2012 es utilizar 
algoritmos de aprendizaje automático para identificar si 
distintos items con datos de lugares, se refieren al mismo
lugar o no, teniendo en cuenta que estas instancias
pueden provenir de fuentes diferentes.

Las instrucciones oficiales del desafio pueden leerse en 
\textit{http://fr.nomao.com/labs/challenge}.

\subsection{Sobre los datos provistos por Nomao}

El conjunto de datos provisto por Nomao se encuentra en
https://archive.ics.uci.edu/ml/datasets/Nomao. 

Nomao no presenta los datos crudos como estan almacenados en la 
base de datos ni como fueron ingresado por los usario, sino que  
cada instancia representa una comparación de 
2 lugares. Los datos originales son transformados y representados
en 118 variables, de las cuales 89 son continuas y 29 son
nominales. Ademas se entrega una variable adicional de identificación (id) y
otra con la clase, que identifica si ambos lugares referencian a un mismo
destino o no. 

El dataset contiene unas 34.465 instancias con un 28\% de datos faltantes.
Los datos faltantes se debe a las limitaciones de cada fuente de datos. Por
ejemplo, cuando se ingresa una dirección manualmente, el usuario no tiene
capacidad de ingresar la información de GPS.

La tabla \ref{table:data_set} en el apendice \ref{appendix1} detalla todas las variables entregadas por 
Nomao, como su tipo de datos. Todas las variables reales, van en el rango de 0 a 1. Las variables
categóricas pueden tener 3 posibles valores: \textit{'n'}, \textit{'s'} 
o \textit{'m'}. Ni la organización del desafio ni Nomao especifícan 
el significado de las variable del dataset ni de los valores
que estas pueden tomar (no se sabe que significa 'n', 's' o 'm').


\section{Métodos aplicados}

El objetivo del Challenge es clasificar correctamente si 2 lugares
referencian al mismo destino. Para lograr lo que básicamente se hace es:
\begin{itemize}
\item Análisis, limpieza y pre-procesamiento datos. 
\item Un análisis de componentes principales para entender mayor el problema. 
\item Análisis de discriminante lineal de Fisher.
\item Análisis cuadratico de Fisher.
\item Análisis de Maquinas de Vector Soporte (Support Vector Machine).
\item Ensamble con los mejores clasificadores.
\end{itemize}

Todos el procesamiento y análisis se hizo usando algoritmos en Python 3. Se
utilizaron como soporte las siguientes librerías:
\begin{itemize}
\item Pandas\footnote{http://pandas.pydata.org/}: Para usar la estructura de datos Data Frame
\item NumPy\footnote{http://www.numpy.org/}: Para operaciones vectoriales
\item Scikit Learn\footnote{http://scikit-learn.org/}: Implementaciones de Análisis discriminante lineal
	y cuadrático de Fisher, Maquinas de Vector Soporte y Análisis
	de componentes principales
\item Matplotlib: Herramienta de graficación en Python
\end{itemize}

Todo los códigos fuentes se encuentran en el siguiente repositorio
\textit{GitHub: https://github.com/gmoncarz/nomao-challenge}

\subsection{Análisis de datos, Limpieza y pre-procesamiento}
El dataset no presenta outliers. Esto se debe a que todas las variables continuas estan
en el rango de datos especificados: entre 0 y 1. 
Esto hace que no sea necesario eliminar ninguna
instancia de datos.

Las variables categóricas también respetan el standard: no hay ninguna
de estas variable que contenga un valor no especificado. Estas 
variable se convirtieron en variables dummies, con el objetivo 
por aplicar métodos que requieran variables númericas.

Todas las variables continuas, excepto las que comienzan con el
nombre \textit{clean\_name}, tienen datos faltantes. Como el rango
de estas variables es de 0 a 1, todas aquellas que tienen datos
faltantes se las reemplazo por el valor -1. No hay una justificación
teórica de por que se escoge el valor -1, pero los clasificadores
respondieron efectivamente a este valor.

\subsection{Análisis de Componentes Principales}
Se corre un análisis de Componentes principales. Se puede correr sobre
todas las variables categóricas ya que estas se convirtieron en 
variables dummies.

\subsection{Cross Validation}
Los métodos que se corren en las sección seguientes pueden tener overfitting. Para
tener resultados precisos evitando lo más posible el overfitting, todos los
métodos que se corren y se detallan en las secciones posteriores se realizan
aplicando Cross Validation de 5 folders.

\subsection{Análisis Discriminante Lineal}
Se corre un análisis de discriminante lineal de Fisher. Este proceso no 
tiene parámetros especiales a configurar.

\subsection{Análisis Discriminante Cuadrático}
Se corre un análisis de discriminante cuadrático de Fisher. Este proceso
no tiene variables especiales a configurar.

\subsection{Maquinas de vector soporte}
Las máquinas de vector soporte pueden discriminar instancias con distintos
tipos de kernel, dependiendo de la naturaleza de los datos de entradas. Como
la naturaleza de los datos es desconocido, y parte del desafio es identificarla,
en este trabajo se realiza un análisis de los 4 kernels más conocidos: lineal,
polinómico, sigmoide y RBF (Radial Basis Function).

A su vez, la eficacia en la clasificación de cada kernel depende de los
parámetros en que el clasificador es entrenado. Estos son valores empíricos
que dependen exclusivamente de cada problema en particular. Es por eso
que corren varios entrenamientos con distintos kernels y distintos parámetros.
La table \ref{table:svm_config} especifica todas las variaciones de maquinas 
de vector soporte corridas.

\begin{table}[ht!]
\caption{Distintas configuraciones de SVM ejecutadas}
\label{table:svm_config}
\centering
\begin{tabular}{l | l l l }
Número & Kernel & Gamma & Degree  \\
\hline
1 & lineal &  &  \\ 
2 & rbf  & 1  &  \\ 
3 & rbf  & 0 &  \\ 
4 & rbf  & 0,1 &  \\ 
5 & rbf  & 0,01 &  \\ 
6 & rbf  & 0,001 &  \\ 
7 & rbf  & 0,0001 &  \\ 
8 & rbf  & 0,00001 &  \\ 
9 & polinómico & 1  & 2 \\ 
10 & polinómico & 0  & 2 \\ 
11 & polinómico & 0,1  & 2 \\ 
12 & polinómico & 0,01  & 2 \\ 
13 & polinómico & 0,001  & 2 \\ 
14 & polinómico & 0,0001  & 2 \\ 
15 & polinómico & 0,00001  & 2 \\ 
16 & polinómico & 1  & 3 \\ 
17 & polinómico & 0  & 3 \\ 
18 & polinómico & 0,1  & 3 \\ 
19 & polinómico & 0,01  & 3 \\ 
20 & polinómico & 0,001  & 3 \\ 
21 & polinómico & 0,0001  & 3 \\ 
22 & polinómico & 0,00001  & 3 \\ 
23 & polinómico & 1  & 4 \\ 
24 & polinómico & 0  & 4 \\ 
25 & polinómico & 0,1  & 4 \\ 
26 & polinómico & 0,01  & 4 \\ 
27 & polinómico & 0,001  & 4 \\ 
28 & polinómico & 0,0001  & 4 \\ 
29 & polinómico & 0,00001  & 4 \\ 
30 & polinómico & 1  & 5 \\ 
31 & polinómico & 0  & 5 \\ 
32 & polinómico & 0,1  & 5 \\ 
33 & polinómico & 0,01  & 5 \\ 
34 & polinómico & 0,001  & 5 \\ 
35 & polinómico & 0,0001  & 5 \\ 
36 & polinómico & 0,00001  & 5 \\ 
37 & sigmoide & 1 &  \\ 
38 & sigmoide & 0 &  \\ 
39 & sigmoide & 0,1 &  \\ 
40 & sigmoide & 0,01 &  \\ 
41 & sigmoide & 0,001 &  \\ 
42 & sigmoide & 0,0001 &  \\ 
43 & sigmoide & 0,00001 &  \\ 
\end{tabular}
\end{table}

\subsection{K-Vecinos mas cercanos}
Se realiza 18 iteraciones de K-Vecinos mas cercanos, iterando entre k=3
a k=20. Se utiliza la distancia de Minkowski.

\subsection{Ensamble}
Este método realiza un ensamble por votación de los siguientes 3
clasificadores corridos previamente:

\begin{itemize}
\item Discriminante lineal de Fisher
\item Máquina de vector soporte polinómica de grado 3 y gamma=0,1. 
\item K Vecinos mas cercanos con K=3
\end{itemize}

\section{Resultados}

La tabla \ref{table:resultados} muestra la eficacia en la clasificacion
y los tiempos de ejecución de cada uno de los algoritmos 

\section{Discusión}

\section{Conclusiones}

bla bla



\appendices

\section{Especificación de los datos provistos por Nomao}
\label{appendix1}
Todas las variables categóricas pueden tener 3 valores posibles:
\textit{'n'}, \textit{'s'} o \textit{'m'}.
Todas las variables continuas son reales con valores entre 
0 y 1. Las variables continuas pueden tener valores faltantes.

\begin{table}[ht!]
\caption{Descripcion de las 118 variables} 
\label{table:data_set}
\begin{tabular}{l | l l }
Numero & Nombre & Tipo \\
       &        &      \\
\hline
1	& id  & string  \\
2	& clean\_name\_intersect\_min  &   real  \\
3	& clean\_name\_intersect\_max  &   real  \\
4	& clean\_name\_levenshtein\_sim  &   real  \\
5	& clean\_name\_trigram\_sim  &   real  \\
6	& clean\_name\_levenshtein\_term  &   real  \\
7	& clean\_name\_trigram\_term  &   real  \\
8	& clean\_name\_including  &    categorica   \\
9	& clean\_name\_equality  &    categorica   \\
10	& city\_intersect\_min  &   real  \\
11	& city\_intersect\_max  &   real \\
12	& city\_levenshtein\_sim  &   real  \\
13	& city\_trigram\_sim  &   real  \\
14	& city\_levenshtein\_term  &   real  \\
15	& city\_trigram\_term  &   real  \\
16	& city\_including  &    categorica   \\
17	& city\_equality  &    categorica   \\
18	& zip\_intersect\_min  &   real  \\
19	& zip\_intersect\_max  &   real  \\
20	& zip\_levenshtein\_sim  &   real  \\
21	& zip\_trigram\_sim  &   real  \\
22	& zip\_levenshtein\_term  &   real  \\
23	& zip\_trigram\_term  &   real  \\
24	& zip\_including  &    categorica   \\
25	& zip\_equality  &    categorica   \\
26	& street\_intersect\_min  &   real  \\
27	& street\_intersect\_max  &   real  \\
28	& street\_levenshtein\_sim  &   real  \\
29	& street\_trigram\_sim  &   real  \\
30	& street\_levenshtein\_term  &   real  \\
31	& street\_trigram\_term  &   real  \\
32	& street\_including  &    categorica   \\
33	& street\_equality  &    categorica   \\
34	& website\_intersect\_min  &   real  \\
35	& website\_intersect\_max  &   real  \\
36	& website\_levenshtein\_sim  &   real  \\
37	& website\_trigram\_sim  &   real  \\
38	& website\_levenshtein\_term  &   real  \\
39	& website\_trigram\_term  &   real  \\
40	& website\_including  &    categorica   \\
41	& website\_equality  &    categorica   \\
42	& countryname\_intersect\_min  &   real  \\
43	& countryname\_intersect\_max  &   real  \\
44	& countryname\_levenshtein\_sim  &   real  \\
45	& countryname\_trigram\_sim  &   real  \\
46	& countryname\_levenshtein\_term  &   real  \\
47	& countryname\_trigram\_term  &   real  \\
48	& countryname\_including  &    categorica   \\
49	& countryname\_equality  &    categorica   \\
50	& geocoderlocalityname\_intersect\_min  &   real  \\
51	& geocoderlocalityname\_intersect\_max  &   real  \\
52	& geocoderlocalityname\_levenshtein\_sim  &   real  \\
53	& geocoderlocalityname\_trigram\_sim  &   real  \\
54	& geocoderlocalityname\_levenshtein\_term  &   real  \\
55	& geocoderlocalityname\_trigram\_term  &   real  \\
56	& geocoderlocalityname\_including  &    categorica   \\
57	& geocoderlocalityname\_equality  &    categorica   \\
58	& geocoderinputaddress\_intersect\_min  &   real  \\
59	& geocoderinputaddress\_intersect\_max  &   real  \\
60	& geocoderinputaddress\_levenshtein\_sim  &   real  \\
61	& geocoderinputaddress\_trigram\_sim  &   real  \\
62	& geocoderinputaddress\_levenshtein\_term  &   real  \\
63	& geocoderinputaddress\_trigram\_term  &   real  \\
64	& geocoderinputaddress\_including  &    categorica   \\
65	& geocoderinputaddress\_equality  &    categorica   \\
66	& geocoderoutputaddress\_intersect\_min  &   real  \\
67	& geocoderoutputaddress\_intersect\_max  &   real  \\
68	& geocoderoutputaddress\_levenshtein\_sim  &   real  \\
69	& geocoderoutputaddress\_trigram\_sim  &   real  \\
70	& geocoderoutputaddress\_levenshtein\_term  &   real  \\
\end{tabular}
\end{table}

\begin{table}[ht!]
\centering
\begin{tabular}{l | l l l}
Numero & Nombre & Tipo & Rango \\
       &        &      &                 \\
\hline
71	& geocoderoutputaddress\_trigram\_term  &   real  \\
72	& geocoderoutputaddress\_including  &    categorica   \\
73	& geocoderoutputaddress\_equality  &    categorica   \\
74	& geocoderpostalcodenumber\_intersect\_min  &   real  \\
75	& geocoderpostalcodenumber\_intersect\_max  &   real  \\
76	& geocoderpostalcodenumber\_levenshtein\_sim  &   real  \\
77	& geocoderpostalcodenumber\_trigram\_sim  &   real  \\
78	& geocoderpostalcodenumber\_levenshtein\_term  &   real  \\
79	& geocoderpostalcodenumber\_trigram\_term  &   real  \\
80	& geocoderpostalcodenumber\_including  &    categorica   \\
81	& geocoderpostalcodenumber\_equality  &    categorica   \\
82	& geocodercountrynamecode\_intersect\_min  &   real  \\
83	& geocodercountrynamecode\_intersect\_max  &   real  \\
84	& geocodercountrynamecode\_levenshtein\_sim  &   real  \\
85	& geocodercountrynamecode\_trigram\_sim  &   real  \\
86	& geocodercountrynamecode\_levenshtein\_term  &   real  \\
87	& geocodercountrynamecode\_trigram\_term  &   real  \\
88	& geocodercountrynamecode\_including  &    categorica   \\
89	& geocodercountrynamecode\_equality  &    categorica   \\
90	& phone\_diff  &   real  \\
91	& phone\_levenshtein  &   real  \\
92	& phone\_trigram  &   real  \\
93	& phone\_equality  &    categorica   \\
94	& fax\_diff  &   real  \\
95	& fax\_levenshtein  &   real  \\
96	& fax\_trigram  &   real  \\
97	& fax\_equality  &    categorica   \\
98	& street\_number\_diff  &   real  \\
99	& street\_number\_levenshtein  &   real  \\
100	& street\_number\_trigram  &   real  \\
101	& street\_number\_equality  &    categorica   \\
102	& geocode\_coordinates\_long\_diff  &   real  \\
103	& geocode\_coordinates\_long\_levenshtein  &   real  \\
104	& geocode\_coordinates\_long\_trigram  &   real  \\
105	& geocode\_coordinates\_long\_equality  &    categorica   \\
106	& geocode\_coordinates\_lat\_diff  &   real  \\
107	& geocode\_coordinates\_lat\_levenshtein  &   real  \\
108	& geocode\_coordinates\_lat\_trigram  &   real  \\
109	& geocode\_coordinates\_lat\_equality  &    categorica   \\
110	& coordinates\_long\_diff  &   real  \\
111	& coordinates\_long\_levenshtein  &   real  \\
112	& coordinates\_long\_trigram  &   real  \\
113	& coordinates\_long\_equality  &    categorica   \\
114	& coordinates\_lat\_diff  &   real  \\
115	& coordinates\_lat\_levenshtein  &   real  \\
116	& coordinates\_lat\_trigram  &   real  \\
117	& coordinates\_lat\_equality  &    categorica   \\
118	& geocode\_coordinates\_diff  &   real  \\
119	& coordinates\_diff  &   real  \\
120	& label (clase) & categorica (1 o 0). \\
\end{tabular}
\end{table}



\end{document}

